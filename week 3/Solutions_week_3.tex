%%%%%%%%%%%%%%%%%%%%%%%%%%%%%%%%%%%%%%%%%%%%%%
% Header
\documentclass[11pt]{report}
\usepackage[english]{babel}
\usepackage[utf8x]{inputenc}
\PassOptionsToPackage{hyphens}{url}\usepackage{hyperref}
\usepackage{graphicx}
\usepackage{fullpage}
\usepackage{nicefrac}
\usepackage[lastexercise]{exercise}
\usepackage[dvipsnames]{xcolor}
\usepackage{listings}

\setlength{\parindent}{0cm}

\renewcommand{\ExerciseHeader}{\large\textbf{\ExerciseName~\ExerciseHeaderNB} - \textbf{\ExerciseTitle}\medskip}

\renewcommand{\ExePartHeader}{\medskip\textbf{\ExePartName\ExePartHeaderNB\ExePartHeaderTitle\medskip}}

\begin{document}
%%%%%%%%%%%%%%%%%%%%%%%%%%%%%%%%%%%%%%%%%%%%%%
\subsubsection*{EMAT10007 -- Introduction to Computer Programming}
\subsection*{\Large Exercises -- Week 3. Loops and Data Structures}

\subsection*{\Large Part 1. Loops}
To complete this week's exercises you will also need to download the {\tt SumFirstTenIntegers.py} and {\tt HowManySquares.py} files from Blackboard.

\begin{Exercise}[title=For Loops]

	\Question{Running the code shown will print letters from the string one by one, displaying the string vertically.}
	\Question{{\tt SumFirstTenIntegers.py} sums even integers from 1 to 10 and prints out the result (30).}
	\Question{We use the value 11 here to ensure that the value 10 is included, as {\tt range(10)} only runs from 0 to 9.}
	\Question{{\tt range(1, 11, 2)} will return odd numbers from 1 to 10.}
	\Question{See file: Solution\_Week3\_Ex1.py
	}
	\Question{(*) See file: Solution\_Week3\_Ex1.py}
\end{Exercise}

\begin{Exercise}[title=While Loops]
	\Question{
	{\tt Words = "Hello World"}\\
	{\tt TargetLetter = 'e'}\\
	{\tt i = 0}\\
	{\tt while Words[i] != TargetLetter:}\\
	{\tt \hspace*{2em} i += 1}\\
	{\tt print("Target letter is at position", i)}
	\vspace{0.5em}

We increment the i to iterate through letters in our {\tt Words} string.
	}
	\Question{(*) How would you change the code to print all the occurrences of the letter?
	
	\textbf{Hint:} What type of loop would be best?}
	\Question{Finding characters or substrings in a string is very useful and so Python has a built-in function {\tt str.find()}. Test yoor program is right by comparing with the {\tt str.find()} function for some different target letters and words.
	
	\textbf{Hint:} Remember the {\tt help()} function.}
	\Question{Open and run the {\tt HowManySquares.py} program. Can you fix the code to use {\tt CamelCase} and the {\tt +=} operator?}
	\Question{In the {\tt HowManySquares.py} example we do not know when the cumulative sum will exceed $1,000,000$. Therefore, we use a {\tt while} loop until the sum exceeds this limit, and therefore the condition no longer evaluates to {\tt True}. Can you add a line that shows the results for every step of the {\tt while} loop?}
	\Question{(*) Using the {\tt break} keyword, can you rewrite Q\ref{Q:whilestring} using a {\tt for} loop, an {\tt if} clause and {\tt enumerate()}?}
\end{Exercise}

\begin{Exercise}[title=More Loops]

	Use loops and conditionals to solve the following problems:
	\Question{Write a program that assigns two variables, say {\tt A} and {\tt B}, to the throws of two random dice. Your program should keep reassigning dice throws to {\tt A} and {\tt B} until {\tt A == B}. Then, create another variable to count the number of times the program assigns A and B random values until both numbers are equal. When this happens, print out a success message and the number of assignments it took for {\tt A == B}.
	
	\textbf{Hint:} Recall you replicated dice throws for an exercise in Week 1.}
    \Question{Implement the ``Number Guessing Game'', which works in the following way:
    \begin{itemize}
        \item Pick a random number between 1 and 100.
        \item Ask the user to guess the number using the {\tt input()} function.
        \item Tell the user if they are correct and stop the program. If they are incorrect, tell them whether their guess is too low or too high.
        \item Repeat until the user has guessed correctly.
        \item Congratulate the user on guessing the number and tell them how many guesses it took them.
    \end{itemize}
	What about when the user guesses a number out of range? Add a check to your program to instruct the user to enter a guess within the accepted range.}  
\end{Exercise}

\pagebreak

\subsection*{\Large Part 2. Data structures}

\begin{Exercise}[title=Lists]
	\Question{Make two lists containing the values {\tt [1,2]} and {\tt [3,4]}.}
	\Question{Change the value 1 to the value 5.}
	\Question{Sort the lists.
	
	\textbf{Hint:} Have a look at {\tt help(list)}.}
	\Question{Make a nested list that contains both lists.}
	\Question{Use two loops to print out all the values in the nested list (2x2 matrix) one by one.}
	\Question{Write a program that asks the user to input a list of 10 words (strings) and then creates a list containing the length of each word. Print out each word and word length, like so:
	
	\vspace{0.5em}
	{\tt Word: Algorithm - Word length: 9}
	\vspace{0.5em}
	
	\textbf{Hint:} You can read all $10$ words at a time, as one large string, and use the {\tt .split()} function on the string. Read the documentation if you are unfamiliar with the {\tt split} function. Then loop through the resulting list of words and print out the length of each word.}
	\Question{(*) List comprehension is an elegant way to produce lists using loops and conditional statements. Read how to do this here: \url{https://www.pythonforbeginners.com/basics/list-comprehensions-in-python}
	
	Using list comprehension, create lists of the following between 0 and 100:
	\begin{itemize}
	    \item odd numbers
	    \item multiples of 3
	    \item prime numbers (extra tricky)
	\end{itemize}
	}
\end{Exercise}

\begin{Exercise}[title=Tuples]
	\Question{Have a look at {\tt help(tuple)}.}
	\Question{Make a tuple named {\tt FondueIngredients} containing the values ``gruyere'' and ``vacherin''.}
	\Question{Print all the items in the tuple.}
	\Question{Change the value ``gruyere'' to the value ``cheddar''. Does it work? Why?
	
	\textbf{Note:} Fondue recipes are sacred.}
	\Question{Is there a function to remove the last item of the tuple? How else could you do it?}
\end{Exercise}

\begin{Exercise}[title=Sets]
	\Question{Have a look at {\tt help(set)}.}
	\Question{Make two set {\tt s1 = \{1,2,5,5,8\}} and {\tt s2 = \{1,2,4,9,2\}}. Print out the sets, are there any duplicates?}
	\Question{Can you access an element of the set based on index e.g. {\tt s1[2]}?}
	\Question{Use the keyword in to check if 4 is in both sets.}
	\Question{Use the operators {\tt \&}, {\tt |}, {\tt -}, {\tt \^{}}. What do they do?}
	\Question{Remove the value 1 from the first set, and add the value 6.}
\end{Exercise}

\begin{Exercise}[title=Dictionaries]
	\Question{Have a look at {\tt help(dict)}.}
	\Question{Make a dictionary that contains {\tt \{"Jill":21, "Sally”:20, "Bob”:20, "Harry”:21\}}. Remember to give it a sensible name in {\tt CamelCase}.}
	\Question{Print out all the keys in the dictionary. Use {\tt help(dict)} to work out how to do this.}
	\Question{Add the item {\tt "Rachel":19} to the dictionary.}
	\Question{Remove the item {\tt "Bob"}.}
	\Question{Add the item {\tt "Jill":22} to the dictionary. Are there two Jills now?}
	\Question{Check if {\tt "Harry"} is in the dictionary.}
\end{Exercise}


\begin{Exercise}[title=FizzBuzz Game (*)]

 	In the game FizzBuzz, we count from 1 to $n$, replacing any multiple of $3$ with the word ``Fizz'' and any multiple of $5$ with the word ``Buzz'' As follows:
 	
 	\centering
 	\vspace{0.5em}
	``$1$, $2$, Fizz, $4$, Buzz, Fizz, $7$, $8$, Fizz, Buzz, $11$, Fizz, $13$, $14$, FizzBuzz, $...$''.
	\vspace{0.5em}
	
	\Question{Create two variables {\tt Mult3} and {\tt Mult5}, setting their values to be the strings {\tt "Fizz"} and {\tt "Buzz"}, respectively.}
	\Question{Create an additional variable {\tt Limit}, which will be the number we count up to.}
	\Question{At the beginning of your program, after you have assigned {\tt Mult3} and {\tt Mult5} their values, you will need to ask the user to \textbf{input} a value for {\tt Limit}. This can be done using the {\tt input()} function, which waits for the user to input some \emph{string} when you run your program, before continuing.
	
	\textbf{Hint:} you will need to \textbf{convert} the input string created by {\tt input()} into an integer, using {\tt int()}.}
	\Question{The computer should say each number from 1 to {\tt Limit}, replacing each multiple of $3$ with the word ``Fizz'' and each multiple of $5$ with ``Buzz''.
	What kind of \textbf{loop} will you need for this?}
	\Question{You will also need to use the {\tt \%} operator, which returns the remainder of a division e.g. $4~\textrm{mod}~3 = 1$ and $15~\textrm{mod}~3 = 0$, indicating that $15$ is a multiple of $3$. You will need to check whether each number is either a multiple of 3, a multiple of 5, or \emph{both}.
	
	\textbf{Hint:} Start with the basic loop, printing out each number, and then work on replacing it with ``Fizz'', ``Buzz'', or ``FizzBuzz'', in stages.}
\end{Exercise}

\end{document}