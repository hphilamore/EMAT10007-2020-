%%%%%%%%%%%%%%%%%%%%%%%%%%%%%%%%%%%%%%%%%%%%%%
% Hea
\documentclass[11pt]{report}
\usepackage[english]{babel}
\usepackage[utf8x]{inputenc}
\usepackage{amsmath}
\usepackage{hyperref}
\usepackage{graphicx}
\usepackage{fullpage}
\begin{document}
%%%%%%%%%%%%%%%%%%%%%%%%%%%%%%%%%%%%%%%%%%%%%%
\title{Supplementary material - Git}
\subsection*{Introduction to Computer Programming}
\subsection*{Supplementary material - Git}
\begin{enumerate}
	\item \textbf{Introduction}
	\begin{itemize}
		\item {\tt Git} is a form of `\textbf{version}(/source/revision) \textbf{control system}' (VCS), which allows you to manage your files (usually source code or data for programs/documents) by associating revisions to your code with a time stamp and a description of the changes.

		\item Systems like {\tt Git} are particularly useful for source code because small changes can break programs completely, and so it is crucial that we are able to store and reference copies of our source code at key points e.g.\ when we successfully implement new functionality, or before we make (potentially breaking) changes to existing code.
	\end{itemize}

	\item {\bf GitHub}\\
	\href{https://github.com/}{GitHub} is a website that offers hosting of your {\tt Git} repositories and makes using {\tt Git} much more appealing for both new users of {\tt Git} as well as those interested in getting involved with the open-source software community.\\
	It allows you to back up your source files and revision history online, host your code publicly or \emph{privately} (this is important at university, and so is offered for free to students - more on this later) and even allows you to add others to your project as contributors, when working on group or hobby projects. Because copies of your code are stored by GitHub, as well as on your local computer, you have backups in case files become corrupted or hardware lost/damaged.\\
	{\tt Git} and GitHub are not the same thing; the latter is a company which provides hosting and other utility services based around {\tt Git}, the VCS tool.
	\begin{itemize}
		\item You can get familiar with using {\tt Git} by going to the URL below and following the tutorial:\\
		\url{https://try.github.io/levels/1/challenges/1}

		\item GitHub also provides walk-throughs for getting started with {\tt Git}, such as for setting up a new repository, hosting the repository on GitHub and pushing/pulling changes to/from GitHub as you work on your projects.

		\item You should consider making an account on GitHub with your student e-mail address, where you will be provided with free \textbf{private} repositories.
		\begin{itemize}
			\item This is crucial because your projects at university \textbf{should be kept private} from other students while you are working on them before you have completed your submission and it has been fully assessed.
			\item You can still share private repositories with other students if you are working on a group project together, but only those students.
			\item This can be very useful for large projects, particularly for dissertations/final year projects.
		\end{itemize}

		\item GitHub even provides a free \href{https://desktop.github.com/}{GUI utility} for using {\tt Git} if you are not comfortable using the command-line or terminal (available for Windows/Mac). {\tt Git} typically comes installed by default on various Linux distributions.

		\item To get started, you can click `New repository' and follow the instructions to set up a new \textbf{repository} which stores all of your files, as well as the associated revisions, called `commits', related to a project.
	\end{itemize}
\end{enumerate}
\end{document}
