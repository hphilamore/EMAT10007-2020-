%%%%%%%%%%%%%%%%%%%%%%%%%%%%%%%%%%%%%%%%%%%%%%
% Header
\documentclass[11pt]{report}
\usepackage[english]{babel}
\usepackage[utf8x]{inputenc}
\usepackage{hyperref}
\usepackage{graphicx}
\usepackage{fullpage}
\usepackage{amssymb}
\usepackage{tasks}
\usepackage[lastexercise]{exercise}

\setlength{\parindent}{0cm}

\renewcommand{\ExerciseHeader}{\large\textbf{\ExerciseTitle}\medskip}

\renewcommand{\ExePartHeader}{\medskip\textbf{\ExePartName\ExePartHeaderNB\ExePartHeaderTitle\medskip}}

\NewTasks[style=multiplechoice]{choices}[\choice](4)
\NewTasks[style=multiplechoice]{choices_short}[\choice](3)
\NewTasks[style=multiplechoice]{choices_single}[\choice](1)
\newcommand*\correct{\PrintSolutionsTF{\checkedchoicebox}{\choicebox}}


\begin{document}
%%%%%%%%%%%%%%%%%%%%%%%%%%%%%%%%%%%%%%%%%%%%%%
\title{Class Test: Syntax Test}
\subsubsection*{EMAT10007 -- Introduction to Computer Programming}
\subsection*{\Large Syntax Test - Examples}

\subsection*{Overview}
\begin{itemize}
    \item The lab session on \textbf{Friday 1st November} will be the ``Syntax Test'' class test. You will be tested on your understanding of the core principles of programming, with emphasis on the specific syntax used by the Python programming language.

	\item The test will be multiple choice but will only have \textbf{one} correct answer.

    \item If you are wondering ``where are the answers to the exercises?'' - try them out! The best way to revise for this test is to work through the examples and modifying them until they work or give the expected output.
    
    \item Note blank lines are still numbered in the coding snippets.

    \item Please use the Support Forums to ask questions.
\end{itemize}

\begin{Exercise}[title=Practice Exercises]
    \Question{Which line causes an error?
    
    \vspace{1em}
    {\tt 1CatName = "Whiskers"}\\
    {\tt CatName2 = "Felix"}\\
    
    {\tt \# Print the names of two cats}\\
    {\tt print("I have two cats named ", CatName1, " and ", CatName2)}
    \vspace{1em}
    
    \begin{choices}
        \choice Line 1
        \choice Line 2
        \choice Line 4
        \choice Line 5
    \end{choices}
    }
    \Question{Which line causes an error?
    
    \vspace{1em}
    {\tt Counter = 0}\\
    {\tt While Counter < 10:}\\
    {\tt \hspace*{2em}print(Counter)}\\
    {\tt \hspace*{2em}Counter += 1}
    \vspace{1em}
    
    \begin{choices}
        \choice Line 1
        \choice Line 2
        \choice Line 3
        \choice Line 4
    \end{choices}
    }
    \Question{Which line causes an error?
    
    \vspace{1em}
    {\tt A = 10}\\
    {\tt B = 5}\\
    {\tt C = ((A + 2) * (B - 2) / 5}\\
    {\tt print(C)}
    \vspace{1em}

    \begin{choices}
        \choice Line 1
        \choice Line 2
        \choice Line 3
        \choice Line 4
    \end{choices}
    }
    
    \Question{Which line causes an error?
    
    \vspace{1em}
    {\tt Ask for a word}\\
    {\tt Word = input("Enter a word:")}\\
    
    {\tt \# Print the length of the word}\\
    {\tt print(len(Word))}
    \vspace{1em}
    
    \begin{choices}
        \choice Line 1
        \choice Line 2
        \choice Line 4
        \choice Line 5
    \end{choices}
    }
    
    \Question{Which line causes an error?
    
    \vspace{1em}
    {\tt \# Ask for a number}\\
    {\tt Number = input("Enter a number:")}\\
    
    {\tt \# Is the number bigger than 10?}\\
    {\tt print(Number > 10)}
    \vspace{1em}
    
    \begin{choices}
        \choice Line 1
        \choice Line 2
        \choice Line 4
        \choice Line 5
    \end{choices}
    }
    \Question{Which line causes an error?
    
    \vspace{1em}
    {\tt Number = 5}\\
    {\tt if (Number \% 2) == 0}:\\
    {\tt \hspace*{2em} \# Print the number is even}\\
    {\tt \hspace*{2em} print("The number is even.")}\\
    {\tt else:}\\
    {\tt \hspace*{2em} \# Print the number is odd}\\
    {\tt print("The number is odd.")}
    \vspace{1em}
    
    \begin{choices_short}
        \choice Line 2
        \choice Line 3
        \choice Line 4
        \choice Line 5
        \choice Line 6
        \choice Line 7
    \end{choices_short}
    }
    \Question{What type of error is caused by the following code?
    
    \vspace{1em}
    {\tt A = 3}\\
    {\tt B = 4}\\
    {\tt C = math.sqrt((A ** 2) + (B ** 2))}\\
    {\tt print(C)}
    \vspace{1em}
    
    \begin{choices_short}
        \choice {\tt SyntaxError}
        \choice {\tt NameError}
        \choice {\tt TypeError}
    \end{choices_short}
    }
    
    \Question{Consider the following:
    
    \vspace{1em}
    {\tt \# Print out all the multiples of three up to the limit}\\
    {\tt for Value in range(Limit):}\\
    {\tt \hspace*{2em}if (Value \% 3) == 0:}\\
    {\tt \hspace*{4em} print(Value)}
    \vspace{1em}
    
    Which line causes an error?
    
    \begin{choices}
        \choice Line 1
        \choice Line 2
        \choice Line 3
        \choice Line 4
    \end{choices}
    
    
    What type of error is thrown?
    
    \begin{choices_short}
        \choice {\tt SyntaxError}
        \choice {\tt NameError}
        \choice {\tt TypeError}
    \end{choices_short}
    }
    
    \Question{What is the error in the following code?
    
    \vspace{1em}
    {\tt OrigTrilogy = \{"A New Hope", "The Empire Strikes Back", "Return of the Jedi"\}}\\
    {\tt print(OrigTrilogy[-1])}
    \vspace{1em}
    
    \begin{choices_single}
        \choice {\tt IndexError} - Index out of range for the list
        \choice {\tt TypeError} - Tuples are immutable
        \choice {\tt KeyError} - Incorrect key for the dictionary
        \choice {\tt TypeError} - Indexes cannot be used with sets
    \end{choices_single}
    }
    
    \Question{What is the error in the following code?
    
    \vspace{1em}
    {\tt ShoppingItems = ["Bread", "Cookies", "Jam", "Apples"]}\\
    {\tt print(ShoppingItems[4])}
    \vspace{1em}
    
    \begin{choices_single}
        \choice {\tt IndexError} - Index out of range for the list
        \choice {\tt TypeError} - Tuples are immutable
        \choice {\tt KeyError} - Incorrect key for the dictionary
        \choice {\tt TypeError} - Indexes cannot be used with sets
    \end{choices_single}
    }
    
    \Question{What is the error in the following code?
    
    \vspace{1em}
    {\tt LuckyNumbers = (3, 7, 12, 8)}\\
    {\tt LuckyNumbers[0] = 13}
    \vspace{1em}
    
    \begin{choices_single}
        \choice {\tt IndexError} - Index out of range for the list
        \choice {\tt TypeError} - Tuples are immutable
        \choice {\tt KeyError} - Incorrect key for the dictionary
        \choice {\tt TypeError} - Indexes cannot be used with sets
    \end{choices_single}
    }
    
    \Question{Select the correct output to the following:
    
    \vspace{1em}
    {\tt A = True}\\
    {\tt B = False}\\
    {\tt C = A and B}\\
    {\tt print(C)}
    \vspace{1em}
    
    \begin{choices}
        \choice {\tt true}
        \choice {\tt False}
        \choice {\tt True}
        \choice {\tt 0}
    \end{choices}
    }
    
    \Question{Select the correct output to the following:
    
    \vspace{1em}
    {\tt A = True}\\
    {\tt B = False}\\
    {\tt C = A or B}\\
    {\tt print(not C)}
    \vspace{1em}
    
    \begin{choices}
        \choice {\tt true}
        \choice {\tt False}
        \choice {\tt True}
        \choice {\tt 0}
    \end{choices}
    }
    
    \Question{Select the correct output to the following:
    
    \vspace{1em}
    {\tt A = 5}\\
    {\tt B = 2 * A}\\
    {\tt C = A}\\
    {\tt print(C - B)}
    \vspace{1em}
    
    \begin{choices}
        \choice {\tt -A}
        \choice {\tt False}
        \choice {\tt -10}
        \choice {\tt -5}
    \end{choices}
    }
    
    \Question{Select the correct output to the following:
    
    \vspace{1em}
    {\tt A, B = 5, 10}\\
    {\tt C = A * 2}\\
    {\tt print(C == B)}
    \vspace{1em}
    
    \begin{choices}
    \choice {\tt True}
    \choice {\tt False}
    \choice {\tt 5}
    \choice {\tt 10}
    \end{choices}    
    }
    
    \Question{Select the correct output to the following:
    
    \vspace{1em}
    {\tt A = 12}\\
    {\tt C = 7}\\
    {\tt A = B = C}\\
    {\tt print(A, B)}
    \vspace{1em}
    
    \begin{choices}
    \choice {\tt 7 7}
    \choice {\tt C C}
    \choice {\tt 12 12}
    \choice {\tt 12 7}
    \end{choices}    
    }
    \Question{Select the missing operators:
    
    \vspace{1em}
    {\tt NumberList = [10, 12, 15, 20, 24, 27, 30, 35]}\\
    {\tt Total = 0}\\
    
    {\tt for Number in NumberList:}\\
    {\tt \hspace*{2em} if (Number \% 10) == 0:}\\
    {\tt \hspace*{4em} Total <?> 1}\\
    {\tt \hspace*{2em} else:}\\
    {\tt \hspace*{4em} Total <?> 1}\\
    {\tt print(Total)}\\
    
    {\tt -----}\\
    {\tt Output:}\\
    
    {\tt 2}
    \vspace{1em}
    
    \begin{choices}
        \choice {\tt -=, -=}
        \choice {\tt +=, +=}
        \choice {\tt +=, -=}
        \choice {\tt -=, +=}
    \end{choices}
    }
    
    \Question{At the end of each code snippet, what is the {\tt type} of {\tt A}?}
    
    \vspace{1em}
    {\tt A = "22"}
    \vspace{1em}
    
    \begin{choices}
        \choice {\tt string}
        \choice {\tt int}
        \choice {\tt bool}
        \choice {\tt float}
    \end{choices}
    
    \vspace{1em}
    {\tt B = 10}\\
    {\tt A = B / 2}
    \vspace{1em}
    
    \begin{choices}
        \choice {\tt bool}
        \choice {\tt int}
        \choice {\tt complex}
        \choice {\tt float}
    \end{choices}
    
    \vspace{1em}
    {\tt C = 6}\\
    {\tt B = 10}\\
    {\tt A = C != B}
    \vspace{1em}
    
    \begin{choices}
        \choice {\tt string}
        \choice {\tt int}
        \choice {\tt bool}
        \choice {\tt complex}
    \end{choices}
    
    \vspace{1em}
    {\tt C = 1 - 3j}\\
    {\tt B = 2}\\
    {\tt A = C ** B}
    \vspace{1em}
    
    \begin{choices}
        \choice {\tt complex}
        \choice {\tt int}
        \choice {\tt bool}
        \choice {\tt float}
    \end{choices}
    
    \vspace{1em}
    {\tt A = "22"}\\
    {\tt A = float(A)}
    \vspace{1em}
    
    \begin{choices}
        \choice {\tt string}
        \choice {\tt int}
        \choice {\tt complex}
        \choice {\tt float}
    \end{choices}
    
        \vspace{1em}
    {\tt B = 2.5}\\
    {\tt A = 2 / int(B)}
    \vspace{1em}
    
    \begin{choices}
        \choice {\tt string}
        \choice {\tt int}
        \choice {\tt bool}
        \choice {\tt float}
    \end{choices}
    
    \Question{Select the correct index:
    
    \vspace{1em}
    {\tt StarWarsYears = [1977, 1980, 1983, 1999, 2002, 2005, 2015, 2017, 2019]}\\
    {\tt print("The Empire Strikes Back came out in ", StarWarsYears[<?>])}\\
    
    {\tt -----}\\
    {\tt Output:}\\
    
    {\tt The Empire Strikes Back came out in 1980}
    \vspace{1em}
    
    \begin{choices}
        \choice 0
        \choice 1
        \choice 2
        \choice 4
        \choice 5
        \choice 6
        \choice 7
        \choice 8
    \end{choices}
    
    \vspace{1em}
    {\tt EpisodeYears = \{"IV":1977, "V": 1980, "I": 1983\}}\\
    {\tt print("A New Hope came out in ", EpisodeYears[<?>])}\\
    
    {\tt -----}\\
    {\tt Output:}\\
    
    {\tt A New Hope came out in 1977}
    \vspace{1em}
    
    \begin{choices}
        \choice {\tt 4}
        \choice {\tt "IV"}
        \choice {\tt "4"}
        \choice {\tt IV}
    \end{choices}
    }
    
    \Question{Select the correct output:
    
    \vspace{1em}
    {\tt ShoppingItems = ["Bread", "Cookies", "Jam", "Salad", "Apples"]}\\
    {\tt ShoppingItems[-2] = ShoppingItems[1]}\\
    {\tt print(ShoppingItems)}
    \vspace{1em}
    
    \begin{choices_single}
        \choice {\tt ["Bread", "Salad", "Jam", "Cookies", "Apples"]}
        \choice {\tt ["Bread", "Cookies", "Jam", "Cookies", "Apples"]}
        \choice {\tt ["Bread", "Bread", "Jam", "Cookies", "Apples"]}
        \choice {\tt ["Bread", "Cookies", "Jam", "Bread", "Apples"]}
    \end{choices_single}
    }
    
    \Question{Select the correct output:
    
    \vspace{1em}
    {\tt Word = "encouragement"}\\
    {\tt Total = 0}
    {\tt for Letter in Word:}\\
    {\tt \hspace*{2em} if Letter != "e":}\\
    {\tt \hspace*{4em} Total += 1}\\
    {\tt \hspace*{2em} elif Letter == "g":}\\
    {\tt \hspace*{4em} Total -= 1}\\
    {\tt print(Total)}\\
    \vspace{1em}
    
    \begin{choices}
        \choice {\tt 9}
        \choice {\tt 2}
        \choice {\tt 3}
        \choice {\tt 10}
    \end{choices}
    }
    
    \Question{Select the correct output:
    
    \vspace{1em}
    {\tt NumberList = []}\\
    {\tt for Number in range(1, 6):}\\
    {\tt \hspace*{2em} NumberList.append(Number ** 3)}\\
    {\tt print(NumberList)}\\
    \vspace{1em}
    
    \begin{choices_single}
        \choice {\tt [0, 1, 8, 27, 64]}
        \choice {\tt [0, 1, 2, 3, 4, 5]}
        \choice {\tt [1, 2, 3, 4, 5]}
        \choice {\tt [1, 8, 27, 64, 125]}
    \end{choices_single}
    }
    
    \Question{Select the correct output:
    
    \vspace{1em}
    {\tt NumberList = []}\\
    {\tt for Number in range(12):}\\
    {\tt \hspace*{2em} NumberList.append(Number // 2)}\\
    {\tt print(NumberList.pop())}\\
    \vspace{1em}
    
    \begin{choices}
        \choice {\tt 5.5}
        \choice {\tt 6.5}
        \choice {\tt 5}
        \choice {\tt 6}
    \end{choices}
    }
    
    \Question{Select the missing functions:
    
    \vspace{1em}
    {\tt WordList = []}\\
    {\tt Sentence = "It was a nice day."}\\
    
    {\tt \# Add each word from the sentence into a list}\\
    {\tt for Word in Sentence.<?>(" "):}\\
    {\tt \hspace*{2em} WordList.<?>(Word)}\\
    {\tt print(WordList)}\\
    
    {\tt -----}\\
    {\tt Output: }\\
    
    {\tt ["It", "was", "a", "nice", "day."]}
    \vspace{1em}
    
    \begin{choices}
        \choice {\tt separate, add}
        \choice {\tt split, add}
        \choice {\tt split, append}
        \choice {\tt separate, append}
    \end{choices}
    }
    \Question{Select the missing function and values:
    
    \vspace{1em}
    {\tt ProduceStock = \{"Apples":5, "Bananas": 2, "Oranges": 10, "Pears": 4\}}\\
    {\tt OrderList = []}\\
    
    {\tt \# Print to order more of an item if low stock}\\
    {\tt for Item, Amount in ProduceStock.<?>():}\\
    {\tt \hspace*{2em} if Amount < <?>:}\\
    {\tt \hspace*{4em} OrderList.append(<?>)}\\
    {\tt print("Order more: ", OrderList)}
    
    {\tt -----}\\
    {\tt Output: }\\
    
    {\tt Order more: ["Bananas", "Pears"]}\\
    \vspace{1em}

    \begin{choices_short}
        \choice {\tt keys, 5, Item}
        \choice {\tt items, 4, Amount}
        \choice {\tt items, 5, Item}
        \choice {\tt keys, 4, Amount}
        \choice {\tt keys, 4, Item}
        \choice {\tt items, 5, Amount}
    \end{choices_short}
    }
    
    \Question{Complete the following code:
    
    \vspace{1em}
    {\tt ProduceStock = \{"Apples":5, "Bananas": 2, "Oranges": 0, "Pears": 4\}}\\
    
    {\tt \# Do we have item in stock?}\\
    {\tt RequestedItem = input("Which item would you like?")}\\
    
    {\tt if RequestedItem <?> ProduceStock:}\\
    {\tt \hspace*{2em} if ProduceStock[RequestedItem] <?> 0:}\\
    {\tt \hspace*{4em} print("Yes we have your item")}\\
    {\tt \hspace*{2em} else:}\\
    {\tt \hspace*{4em} print("Sorry, we have run out of your item")}
    
    \vspace{1em}

    \begin{choices}
        \choice {\tt ==, in}
        \choice {\tt not, ==}
        \choice {\tt in, !=}
        \choice {\tt ==, !=}
    \end{choices}
    }
    
    \Question{Complete the following code:
    
    \vspace{1em}
    {\tt ProduceStock = \{"Apples":5, "Bananas": 2, "Oranges": 10, "Pears": 4\}}\\
    
    {\tt \# Do we sell an item?}\\
    {\tt RequestedItem = input("Which item would you like?")}\\
    
    {\tt \# Add the item to the dictionary}\\
    {\tt if RequestedItem <?> ProduceStock:}\\
    {\tt \hspace*{2em} print("Sorry we do not currently sell that item")}\\
    {\tt \hspace*{2em} <?>}\\
    {\tt else:}\\
    {\tt \hspace*{2em} print("Yes, we sell that item.")}
    
    \vspace{1em}

    \begin{choices_single}
        \choice {\tt ==, ProduceStock.append(RequestedItem)}
        \choice {\tt !=, ProduceStock[RequestedItem] = 0}
        \choice {\tt in, ProduceStock + RequestedItem}
        \choice {\tt not in, ProduceStock[RequestedItem] = 0}
        \choice {\tt not in, ProduceStock.append(RequestedItem)}
        \choice {\tt in, ProduceStock + RequestedItem}
    \end{choices_single}
    }
    
\end{Exercise}

\end{document}
