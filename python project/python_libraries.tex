%%%%%%%%%%%%%%%%%%%%%%%%%%%%%%%%%%%%%%%%%%%%%%
% Header
\documentclass[12pt]{article}
\usepackage[english]{babel}
\usepackage[utf8x]{inputenc}
\usepackage{hyperref}
\usepackage{graphicx}
\usepackage{fullpage}
\usepackage{enumitem}
\usepackage{color,soul}

\title{External Python Libraries}
\date{}
\begin{document}
\maketitle

\noindent Here are some suggestions of external libraries that could be used in Assignment 2:
\\
\\
\textbf{PyGame:} Used to make games and other multi-media applications. It can make graphics and play sounds.
\\
\\
\textbf{Tkinter:} Python's standard library for making Graphical User Interfaces (GUIs).
\\
\\
\textbf{Matplotlib:} Used to make 2-dimensional graphs and plots e.g line plots, histograms, power spectra, bar charts, errorcharts, scatterplots, etc. 
\\
\\
\textbf{NumPy:} Used to process multi-dimensional arrays and matrices. It has a wide variety of mathematical functions and can be useful to analyse data. 
\\
\\
\textbf{BeautifulSoup:} A HTML parser used for web scraping.
\\
\\
\textbf{PySpice:} Provides a Python interface to the Ngspice and Xyce circuit simulators.
\\
\\
\textbf{Pandas:} Used for data manipulation and analysis. In particular, it offers data structures and operations for manipulating numerical tables and time series. 
\\
\\
\textbf{PySimpleGui:} An alternative library for making GUIs.
\\
\\
\textbf{Seaborn: } Used for data visualisation and is good for drawing attractive and informative statistical graphics. (Based on Matplotlib and works very well with Pandas.)
\\
\\
\textbf{SciPy:} Used for both scientific and technical computation. It is very useful for analysising data and solving mathematical equations and algorithms. It is also often used for machine learning and image manipulation. Includes modules for graphics and plotting, optimization, integration, special functions, signal and image processing, genetic algorithms, ODE solvers, and others.

\end{document}